\documentclass{article}
\usepackage[spanish]{babel}
\usepackage{multirow}
\usepackage{array}

\title{Resumen de Instrucciones y Registros en MIPS}
\author{}
\date{}

\begin{document}

\maketitle

\section*{Registros en MIPS}

\begin{table}[h]
\centering
\begin{tabular}{|l|l|l|l|}
\hline
\textbf{Nombre} & \textbf{Número} & \textbf{Uso} & \textbf{¿Se conserva?} \\ \hline
\verb|$zero| & 0 & Valor constante 0 & No aplica \\ \hline
\verb|$at| & 1 & Temporal para el ensamblador & No \\ \hline
\verb|$v0-$v1| & 2-3 & Valores de retorno de funciones & No \\ \hline
\verb|$a0-$a3| & 4-7 & Argumentos de funciones & No \\ \hline
\verb|$t0-$t7| & 8-15 & Temporales & No \\ \hline
\verb|$s0-$s7| & 16-23 & Temporales almacenados & Sí \\ \hline
\verb|$t8-$t9| & 24-25 & Temporales adicionales & No \\ \hline
\verb|$k0-$k1| & 26-27 & Reservados para el SO & No \\ \hline
\verb|$gp| & 28 & Puntero global & Sí \\ \hline
\verb|$sp| & 29 & Puntero de pila & Sí \\ \hline
\verb|$fp| & 30 & Puntero de marco & Sí \\ \hline
\verb|$ra| & 31 & Dirección de retorno & Sí \\ \hline
\end{tabular}
\end{table}

\begin{itemize}
    \item \verb|$zero|: Siempre contiene el valor 0.
    \item \verb|$sp| y \verb|$fp|: Gestionan la pila.
    \item \verb|$ra|: Almacena la dirección de retorno en llamadas a procedimientos.
\end{itemize}

\section*{Instrucciones en MIPS}
Las instrucciones se clasifican en varios formatos: \textbf{R} (registro), \textbf{I} (inmediato), y \textbf{J} (salto)

\subsection*{1. Instrucciones Aritméticas y Lógicas}
\begin{itemize}
    \item \textbf{Suma/Resta}:
    \begin{itemize}
        \item \verb|add $rd, $rs, $rt| $\rightarrow$ \verb|$rd = $rs + $rt|
        \item \verb|sub $rd, $rs, $rt| $\rightarrow$ \verb|$rd = $rs - $rt|
        \item \verb|addi $rt, $rs, inm| $\rightarrow$ \verb|$rt = $rs + inm| (con extensión de signo)
    \end{itemize}
    
    \item \textbf{Lógicas}:
    \begin{itemize}
        \item \verb|and $rd, $rs, $rt| $\rightarrow$ AND bit a bit
        \item \verb|ori $rt, $rs, inm| $\rightarrow$ OR con constante (cero extendido)
    \end{itemize}
\end{itemize}

\subsection*{2. Transferencia de Datos}
\begin{itemize}
    \item \textbf{Carga/Almacenamiento}:
    \begin{itemize}
        \item \verb|lw $s1,100($s2)| $\rightarrow$ Carga palabra desde memoria
        \item \verb|sw $s1,100($s2)| $\rightarrow$ Almacena palabra en memoria
        \item \verb|lui $s1, 100| $\rightarrow$ Carga un valor inmediato en los 16 bits superiores
    \end{itemize}
\end{itemize}

\subsection*{3. Saltos Condicionales e Incondicionales}
\begin{itemize}
    \item \textbf{Condicionales}:
    \begin{itemize}
        \item \verb|beq $rs, $rt, etiqueta| $\rightarrow$ Salta si \verb|$rs == $rt|
        \item \verb|bne $rs, $rt, etiqueta| $\rightarrow$ Salta si \verb|$rs != $rt|
        \item \verb|slt $rd, $rs, $rt| $\rightarrow$ \verb|$rd = 1| si \verb|$rs < $rt| (con signo)
    \end{itemize}
    
    \item \textbf{Incondicionales}:
    \begin{itemize}
        \item \verb|j etiqueta| $\rightarrow$ Salto directo
        \item \verb|jal etiqueta| $\rightarrow$ Salto y enlace (para llamadas a funciones)
        \item \verb|jr $rs| $\rightarrow$ Salto a la dirección en \verb|$rs|
    \end{itemize}
\end{itemize}

\subsection*{4. Instrucciones Especiales}
\begin{itemize}
    \item \textbf{División/Multiplicación}:
    \begin{itemize}
        \item \verb|mult $rs, $rt| $\rightarrow$ Resultado en registros \verb|Hi| y \verb|Lo|
        \item \verb|div $rs, $rt| $\rightarrow$ Cociente en \verb|Lo|, resto en \verb|Hi|
    \end{itemize}
    
    \item \textbf{Desplazamiento}:
    \begin{itemize}
        \item \verb|sll $rd, $rt, shamt| $\rightarrow$ Desplazamiento lógico a la izquierda
    \end{itemize}
\end{itemize}

\section*{Formatos de Instrucción}
\begin{enumerate}
    \item \textbf{Tipo R}:
    \begin{itemize}
        \item Campos: cod oper (6) , rs (5) , rt (5) , rd (5) , desplaz (5) ,func (6)
        \item Ejemplo: \verb|add|, \verb|sub|, \verb|and|
    \end{itemize}
    
    \item \textbf{Tipo I}:
    \begin{itemize}
        \item Campos: cod oper (6) , rs (5) , rt (5) , inmediato (16)
        \item Ejemplo: \verb|addi|, \verb|lw|, \verb|beq|
    \end{itemize}
    
    \item \textbf{Tipo J}:
    \begin{itemize}
        \item Campos: cod oper (6) , dirección (26)
        \item Ejemplo: \verb|j|, \verb|jal|
    \end{itemize}
\end{enumerate}

\section*{Convenciones Importantes}
\begin{itemize}
    \item \textbf{Extensión de signo}: En instrucciones como \verb|addi|, el inmediato se extiende a 32 bits.
    \item \textbf{Alineación}: Las palabras en memoria deben estar alineadas a direcciones múltiplos de 4.
    \item \textbf{Pila}: Crece hacia direcciones más bajas; \verb|$sp| apunta al tope.
\end{itemize}

\end{document}